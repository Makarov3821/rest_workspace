\documentclass[a4paper,12pt]{article}
\usepackage[hmargin=1.6cm,vmargin=1.6cm]{geometry}
\usepackage{graphicx}% Include figure files
\usepackage{dcolumn}% Align table columns on decimal point
\usepackage{bm}% bold math
\usepackage{multirow} % for multirow
%\usepackage{forest}
%\usepackage{tikz-qtree,tikz-qtree-compat}
%% The amsthm package provides extended theorem environments. For example \begin{split}
\usepackage{amsthm,amsmath,amsfonts}

% set link for ebook
%\usepackage[dvipdfmx,bookmarks=true,bookmarksnumbered=false,
%            colorlinks,linkcolor=black,
%            citecolor=black,urlcolor=black]{hyperref}
\usepackage[bookmarks=true,bookmarksnumbered=false,
            colorlinks,linkcolor=black,
            citecolor=black,urlcolor=black]{hyperref}

%\usepackage{tikz}

%\nofiles
%opening
\title{Rust-based Electronic-structure Simulation Toolkit (REST)}
\author{REST development team}

\begin{document}

\maketitle



%\tableofcontents

\newpage
\section{Restricted and Unrestricted Hartree-Fock}
Refering to Szabo's book for Restricted and Unrestricted Hartree-Fock formulas 
%Close-shell expression refers to Szabo's book, Equations of 3.124-125, on Pages 140-142; 
%while open-shell expression refers to Pages 210-214
\begin{equation}
	\boldsymbol{F}\boldsymbol{C}=\boldsymbol{S}\boldsymbol{C}\varepsilon
\end{equation}
\textbf{WARNNING:}
\begin{equation}
	\sum_{j}F_{ij}^{\alpha}C_{jk}^{\alpha} = \varepsilon_k\sum_{j}S_{ij}C_{jk}^{\alpha}
\end{equation}
With the notation definition of four center integral $\left( ij|kl \right)$ of
\begin{equation}
	\left( ij|kl \right) = \int d\textbf{r}_1d\textbf{r}_2 \phi_{i}^{*}(1)\phi_{j}(1)r_{12}^{-1}\phi_{k}^{*}(2)\phi_{l}(2),
\end{equation}
the close-shell expression of Fock matrix (Similar to Equations 3.148 and 3.154 on Pages 140 and 141) is 
\begin{equation}
	\begin{split}
	F_{ij}=&H_{ij}^{core}+\sum_{kl}D_{lk}\left[\left( ij|kl \right)-\frac{1}{2}\left(il|kj  \right)\right] \\
	=&\left.H_{ij}^{core}+\sum_{kl}D_{kl}\left[\left( ij|kl \right)-\frac{1}{2}\left(ik|jl  \right)\right]\right|_{\textbf{D} \in \mathbb{R}} \\
    \end{split}
\end{equation}
The density matrix in the close shell is defined by (Equation 3.145 on Page 139)
\begin{equation}
	\begin{split}
		D_{kl}=&2\sum_{a}^{N/2}C_{ka}C_{la}^{*}\\
		=&\left.2\sum_{a}^{N/2}C_{ka}C_{la}=D_{lk}\right|_{\textbf{D}\in \mathbb{R}}\\
    %\textbf{D} = &2\textbf{C}\cdot \textbf{C}^{H}=2\textbf{C}\cdot \textbf{C}^{T}\\
    \end{split}
\end{equation}

The unrestricted expression of Fock matrix (Similar to Equations 3.348 and 3.349 on Page 214) is defined by 
\begin{equation}
	\begin{split}
	    F_{ij}^{\sigma}=&H_{ij}^{core}+\sum_{kl}\left[D_{lk}^{\textit{Tot}}\left( ij|kl \right)-D_{kl}^{\sigma}\left(il|kj  \right)\right] \\
		=&H_{ij}^{core}+\left.\sum_{kl}\left[D_{kl}^{\textit{Tot}}\left( ij|kl \right)
		    -D_{kl}^{\sigma}\left(ik|jl  \right)\right]\right|_{\textbf{D} \in \mathbb{R}} \\
	\end{split}
\end{equation}
The density matrix in the general cases (Similar to Equations 3.342-343 on Pages 213) is defined by 
\begin{equation}
	\begin{split}
		D_{kl}^{\sigma}=&\sum_{a}^{all}W_{a}^{\sigma}C_{ka}^{\sigma}C_{la}^{\sigma*}=\sum_{a}^{N_w}W_{a}^{\sigma}C_{ka}^{\sigma}C_{la}^{\sigma*}
		=\left.\sum_{a}^{N_w}W_{a}^{\sigma}C_{ka}^{\sigma}C_{la}^{\sigma}\right|_{\textbf{C}\in\mathbb{R}}\\
		\textbf{D}^{\sigma} = &\textbf{wC}^{\sigma}\cdot \textbf{C}^{\sigma H}=\textbf{wC}^{\sigma}\cdot \textbf{C}^{\sigma T}\\
		\textbf{D}^{\textit{Tot}} = &\sum_{\sigma}\textbf{D}^{\sigma}=\textbf{D}^{\alpha}+\textbf{D}^{\beta}\\
    \end{split}
\end{equation}
Here, $W_{a}^{\sigma}$ is the electron occupation number of the $a$th orbital in the $\sigma$-spin channel.
$N_w$ is the number of orbitals that have non-zero electronic occupation.

For the coulomb term, the RI expression is:
\begin{equation}
	\begin{split}
		J_{ij}^{\sigma} = &\sum_{kl}D_{kl}^{\sigma}\left( ij|kl \right)\\
		= &\sum_{kl}\sum_{\mu}D_{kl}^{\sigma}M_{ij}^{\mu}M_{kl}^{\mu}\\
		= &\sum_{\mu}M_{ij}^{\mu}\left(\sum_{kl}D_{kl}^{\sigma}M_{kl}^{\mu}  \right)\\
	\end{split}
\end{equation}

For the exchange term, the RI expression is
\begin{equation}
	\begin{split}
		K_{ij}^{\sigma} = &\sum_{kl}D_{kl}^{\sigma}\left( ik|jl \right)\\
		= &\sum_{kl}\sum_{\mu}D_{kl}^{\sigma}M_{ik}^{\mu}M_{jl}^{\mu}\\
		= &\sum_{a}^{all}W_{a}^{\sigma}\sum_{kl}\sum_{\mu}C_{ka}^{\sigma}C_{la}^{\sigma}M_{ik}^{\mu}M_{jl}^{\mu}\\
		= &\sum_{\mu}\sum_{a}^{all}W_{a}^{\sigma}\left(\sum_{k}M_{ik}^{\mu}C_{ka}\right)\left(\sum_{l}M_{jl}^{\mu}C_{la}  \right)\\
		= &\sum_{\mu}\sum_{a}^{N_w}W_{a}^{\sigma}
		\left(\sum_{k}M_{ik}^{\mu}C_{ka}^{\sigma}\right)\left(\sum_{l}M_{jl}^{\mu}C_{la}^{\sigma}  \right)\\
		= &\sum_{a}^{N_w}W_{a}^{\sigma}\sum_{\mu}B_{ia}^{\mu\sigma}B_{ja}^{\mu\sigma}\\
		= &\sum_{\mu}\sum_{a}^{N_w}W_{a}^{\sigma}B_{ia}^{\mu\sigma}B_{ja}^{\mu\sigma}
	\end{split}
\end{equation}
where $N_w$ is the number of orbitals with non-zero electronic occupations.

%The last transformation is achieved due to the Hermitian property of $M_{kl}^{\mu}$ for each $\mu$.
The total energy of Restricted Hartree-Fock is defined by (Equation 3.184 on Page 150)
\begin{equation}
	\begin{split}
		E_0 =&\left<\Psi_0\left|\hat{H}\right|\Psi_0\right>\\
		=&\frac{1}{2}\sum_{i}\sum_{j}D_{ji}\left( H_{ij}^{core}+F_{ij} \right)\\
		=&\left.\frac{1}{2}\sum_{i}\sum_{j}D_{ij}\left( H_{ij}^{core}+F_{ij} \right)\right|_{\textbf{D}\in\mathbb{R}}\\
		=&\left.\frac{1}{2}\sum_{i}\sum_{j}D_{ij}^{*}\left( H_{ij}^{core}+F_{ij} \right)\right|_{\textbf{D}\in\mathbb{C}}\\
	\end{split}
\end{equation}
However, for Unrestricted Hartree-Fock method, the total energy expression is (Exercise 3.40 on Page 215)
\begin{equation}
	\begin{split}
		E_0=&\frac{1}{2}\sum_{i}\sum_{j}\left[D_{ji}^{\textit{tot}}H_{ij}^{core}+D_{ji}^{\alpha}F_{ij}^{\alpha}+D_{ji}^{\beta}F_{ij}^{\beta} \right]\\
		   =&\left.\frac{1}{2}\sum_{i}\sum_{j}\left[D_{ij}^{\textit{tot}}H_{ij}^{core}+D_{ij}^{\alpha}F_{ij}^{\alpha}+D_{ij}^{\beta}F_{ij}^{\beta} \right]
		\right|_{\textbf{D}\in\mathbb{R}}\\
		   =&\left.\frac{1}{2}\sum_{i}\sum_{j}\left[D_{ij}^{\textit{tot}*}H_{ij}^{core}+D_{ij}^{\alpha *}F_{ij}^{\alpha}
		+D_{ij}^{\beta *}F_{ij}^{\beta} \right]\right|_{\textbf{D}\in\mathbb{C}}\\
	\end{split}
\end{equation}

\section{Numerical integration and DFT}
V. Blum et al. Computer Physics Communications 180 (2009) 2175-2196
\begin{equation}
	\begin{split}
		V_{xc,ij} = &\int d^3\textbf{r}\psi_i\textbf{r}\hat{v}_{xc}(\textbf{r})\psi_j(\textbf{r})\\
	\end{split}
\end{equation}

\newpage


\section{The Fock exchange potential in reciprocal space}
The Fock exchange potential is
\begin{equation}
    \begin{split}
        V_{x}(\boldsymbol{r},\boldsymbol{r}')&=
        -e^2\sum_{\boldsymbol{q}m}2w_{\boldsymbol{q}}f_{\boldsymbol{q}m}
        \frac{\phi_{\boldsymbol{q}m}^{*}(\boldsymbol{r}')\phi_{\boldsymbol{q}m}(\boldsymbol{r})}{|\boldsymbol{r}-\boldsymbol{r}'|}\\
    \end{split}
\end{equation}
Here, $\boldsymbol{q}$ is the k point, and therefor $w_{\boldsymbol{q}}$ is the weight of the k-point $\boldsymbol{q}$. $m$ is the band index, and
therefore $f_{\boldsymbol{q}m}$ is the occupational number of the band $m$ in the k-point $\boldsymbol{q}$.

To expand the orbital $\phi_{\boldsymbol{q}m}$ in plane wave, we have
\begin{equation}
    \begin{split}
        \phi_{\boldsymbol{q}m}(\boldsymbol{r})&=\frac{1}{\sqrt{\boldsymbol{\Omega}}}\sum_{\boldsymbol{G}}
        C_{\boldsymbol{q}m}(\boldsymbol{G})e^{i(\boldsymbol{q}+\boldsymbol{G})\cdot \boldsymbol{r}}\\
    \end{split}
\end{equation}
Then the Fock exchange potential evolves 
\begin{equation}
    \begin{split}
        V_{x}(\boldsymbol{r},\boldsymbol{r}')&=-\frac{e^2}{\boldsymbol{\Omega}}\sum_{\boldsymbol{q}m}
        \frac{2w_{\boldsymbol{q}}f_{\boldsymbol{q}m}}{{|\boldsymbol{r}-\boldsymbol{r}'|}}
        \sum_{\boldsymbol{G}\boldsymbol{G}'}
        C_{\boldsymbol{q}m}^{*}(\boldsymbol{G}')e^{-i(\boldsymbol{q}+\boldsymbol{G}')\cdot \boldsymbol{r}'}
        C_{\boldsymbol{q}m}(\boldsymbol{G})e^{i(\boldsymbol{q}+\boldsymbol{G})\cdot \boldsymbol{r}}\\
    \end{split}
\end{equation}
Since, we can do the Fourier transform of the Coulomb operator as:
\begin{equation}
    \int d^3\boldsymbol{r}\frac{1}{|\boldsymbol{r}|}e^{-i\boldsymbol{q}\cdot\boldsymbol{r}}=\frac{4\pi}{|\boldsymbol{q}|^2}
\end{equation}
And the reverse Fourier transform will be:
\begin{equation}
    \begin{split}
        \frac{1}{|\boldsymbol{r}|}&=\frac{1}{(2\pi)^3}\int d^3\boldsymbol{q}\frac{4\pi}{|\boldsymbol{q}|^2}e^{i\boldsymbol{q}\cdot\boldsymbol{r}}
        =\frac{1}{2\pi^2}\int d^3\boldsymbol{q}\frac{1}{|\boldsymbol{q}|^2}e^{i\boldsymbol{q}\cdot\boldsymbol{r}}
        %=4\pi\sum\frac{1}{|\boldsymbol{q}|^2}e^{i\boldsymbol{q}\cdot\boldsymbol{r}}\\
        %&=\frac{1}{2\pi^2}\sum_{\boldsymbol{q}}\delta\boldsymbol{q}\frac{1}{|\boldsymbol{q}|^2}e^{i\boldsymbol{q}\cdot\boldsymbol{r}}\\
        %\int d^3\boldsymbol{q}\frac{4\pi}{|\boldsymbol{q}|^2}e^{i\boldsymbol{q}\cdot\boldsymbol{r}}&=\frac{1}{(2\pi)^3}\frac{1}{|\boldsymbol{r}|}\\
    \end{split}
\end{equation}
Insert this equation into the Fock exchange potential
\begin{equation}
    \begin{split}
        V_{x}(\boldsymbol{r},\boldsymbol{r}')&=-\frac{e^2}{2\pi^2\boldsymbol{\Omega}}
        \sum_{\boldsymbol{q}m}2w_{\boldsymbol{q}}f_{\boldsymbol{q}m}
        \int d^3\boldsymbol{k}
        \frac{1}{{|\boldsymbol{k}|^2}}e^{i\boldsymbol{k}\cdot(\boldsymbol{r}-\boldsymbol{r}')}\\
        &\times \sum_{\boldsymbol{G}\boldsymbol{G}'} C_{\boldsymbol{q}m}^{*}(\boldsymbol{G}')e^{-i(\boldsymbol{q}+\boldsymbol{G}')\cdot \boldsymbol{r}'}
        C_{\boldsymbol{q}m}(\boldsymbol{G})e^{i(\boldsymbol{q}+\boldsymbol{G})\cdot \boldsymbol{r}}\\
        &=-\frac{e^2}{2\pi^2\boldsymbol{\Omega}}
        \int d^3\boldsymbol{k}\sum_{\boldsymbol{G}\boldsymbol{G}'} 
        \frac{1}{{|\boldsymbol{k}|^2}}\sum_{\boldsymbol{q}m}2w_{\boldsymbol{q}}f_{\boldsymbol{q}m}\\
        &\times C_{\boldsymbol{q}m}^{*}(\boldsymbol{G}')e^{-i(\boldsymbol{k}+\boldsymbol{q}+\boldsymbol{G}')\cdot \boldsymbol{r}'}
        C_{\boldsymbol{q}m}(\boldsymbol{G})e^{i(\boldsymbol{k}+\boldsymbol{q}+\boldsymbol{G})\cdot \boldsymbol{r}}\\
    \end{split}
\end{equation}
If we make a change
\begin{equation}
    \begin{split}
        \boldsymbol{k}'&=\boldsymbol{k}+\boldsymbol{q}\\
        \boldsymbol{k}&=\boldsymbol{q}-\boldsymbol{k}'\\
    \end{split}
\end{equation}
Then
\begin{equation}
    \begin{split}
        V_{x}(\boldsymbol{r},\boldsymbol{r}')&=-\frac{e^2}{2\pi^2\boldsymbol{\Omega}}
        \int d^3\boldsymbol{k}'\sum_{\boldsymbol{G}\boldsymbol{G}'} 
        \frac{1}{{|\boldsymbol{q}-\boldsymbol{k}'|^2}}\sum_{\boldsymbol{q}m}2w_{\boldsymbol{q}}f_{\boldsymbol{q}m}\\
        &\times C_{\boldsymbol{q}m}^{*}(\boldsymbol{G}')e^{-i(\boldsymbol{k}'+\boldsymbol{G}')\cdot \boldsymbol{r}'}
        C_{\boldsymbol{q}m}(\boldsymbol{G})e^{i(\boldsymbol{k}'+\boldsymbol{G})\cdot \boldsymbol{r}}\\
        &=\int d^3\boldsymbol{k}\sum_{\boldsymbol{G}\boldsymbol{G}'}
        e^{i(\boldsymbol{k}+\boldsymbol{G})\cdot \boldsymbol{r}}e^{-i(\boldsymbol{k}+\boldsymbol{G}')\cdot \boldsymbol{r}'}\\
        &\times-\frac{e^2}{2\pi^2\boldsymbol{\Omega}}\sum_{\boldsymbol{q}m}2w_{\boldsymbol{q}}f_{\boldsymbol{q}m}
         \frac{C_{\boldsymbol{q}m}^{*}(\boldsymbol{G}')
        C_{\boldsymbol{q}m}(\boldsymbol{G})}{|\boldsymbol{k}-\boldsymbol{q}|^2}\\
        &\textcolor{red}{\textbf{?}}\\
    \end{split}
\end{equation}

%\begin{equation}
%    \begin{split}
%        RF[HSE]&=\frac{4\pi}{r^2}(1-\exp^{-r^2/(4\mu^2)})\\
%        RR[lr]&=
%    \end{split}
%\end{equation}

%\bibliographystyle{plain}
%\bibliography{note}
\section{Laplace transform of opposite-spin MP2}
The opposite-spin component of the second-order correlation energy (PT2) is written as
\begin{equation}
    \begin{split}
		E_{c}^{PT2}&=\frac{1}{N_{\boldsymbol{q}}^3}\sum_{\boldsymbol{\delta k}\boldsymbol{k}\boldsymbol{q}'}
		\sum_{ab}^{occ.}\sum_{nm}^{vir.}
		\frac{\left|\sum_{\mu}L_{an}^{\mu}(\boldsymbol{k},\boldsymbol{q})R_{bm}^{\mu}(\boldsymbol{k}',\boldsymbol{q}')\right|^2}
        {\epsilon_{a\boldsymbol{k}}+\epsilon_{b\boldsymbol{k}'}-\epsilon_{n\boldsymbol{q}}-\epsilon_{m\boldsymbol{q}'}}\\
    \end{split}
\end{equation}
For simplicity, we define $\Delta_{a\boldsymbol{k},b\boldsymbol{k}'}^{n\boldsymbol{q},m\boldsymbol{q}'}=
\epsilon_{n\boldsymbol{q}}+\epsilon_{m\boldsymbol{q}'}-\epsilon_{a\boldsymbol{k}}-\epsilon_{b\boldsymbol{k}'}$.
If we use the Laplace transformation 
\begin{equation}
	\begin{split}
	    \frac {1} {\Delta_{a\boldsymbol{k},b\boldsymbol{k}'}^{n\boldsymbol{q},m\boldsymbol{q}'}}
	    &= \int_0^{\infty}dt e^{-t\Delta_{a\boldsymbol{k},b\boldsymbol{k}'}^{n\boldsymbol{q},m\boldsymbol{q}'}}\\
		&= \sum_{q}^{N_{q}}w_{q} e^{-t_q\Delta_{a\boldsymbol{k},b\boldsymbol{k}'}^{n\boldsymbol{q},m\boldsymbol{q}'}}\\
	\end{split}
\end{equation}
to expand the opposite-spin PT2 correlation energy, we have
\begin{equation}
    \begin{split}
		E_{c}^{PT2}&=-\frac{1}{N_{\boldsymbol{q}}^3}\sum_{\boldsymbol{\delta k}\boldsymbol{k}\boldsymbol{q}'}
		\sum_{q}^{N_q}\sum_{ab}^{occ.}\sum_{nm}^{vir.}w_{q}
		\left|\sum_{\mu}L_{an}^{\mu}(\boldsymbol{k},\boldsymbol{q})R_{bm}^{\mu}(\boldsymbol{k}',\boldsymbol{q}')\right|^2
        e^{-t_q\Delta_{a\boldsymbol{k},b\boldsymbol{k}'}^{n\boldsymbol{q},m\boldsymbol{q}'}}\\
		&=-\frac{1}{N_{\boldsymbol{q}}^3}\sum_{\boldsymbol{\delta k}\boldsymbol{k}\boldsymbol{q}'}
		\sum_{q}^{N_q}\sum_{ab}^{occ.}\sum_{nm}^{vir.}
		\left|\sum_{\mu}w_{q}^{\frac 1 4}L_{an}^{\mu}(\boldsymbol{k},\boldsymbol{q})e^{-\frac{1}{2}t_q(\epsilon_{n\boldsymbol{q}}-\epsilon_{a\boldsymbol{k}})}
		w_{q}^{\frac 1 4}R_{bm}^{\mu}(\boldsymbol{k}',\boldsymbol{q}')e^{-\frac{1}{2}t_q(\epsilon_{m\boldsymbol{q}'}-\epsilon_{b\boldsymbol{k}'})}\right|^2\\
		&=-\frac{1}{N_{\boldsymbol{q}}^3}\sum_{\boldsymbol{\delta k}\boldsymbol{k}\boldsymbol{q}'}
		\sum_{q}^{N_q}\sum_{ab}^{occ.}\sum_{nm}^{vir.}
		\left|\sum_{\mu}\bar{L}_{an}^{\mu}(\boldsymbol{k},\boldsymbol{q})\bar{R}_{bm}^{\mu}(\boldsymbol{k}',\boldsymbol{q}')\right|^2\\
		&=-\frac{1}{N_{\boldsymbol{q}}^3}\sum_{\boldsymbol{\delta k}\boldsymbol{k}\boldsymbol{q}'}
		\sum_{q}^{N_q}\sum_{ab}^{occ.}\sum_{nm}^{vir.}
		\sum_{\mu v}\bar{L}_{an}^{\mu *}(\boldsymbol{k},\boldsymbol{q})\bar{L}_{an}^{v}(\boldsymbol{k},\boldsymbol{q})
		\bar{R}_{bm}^{\mu *}(\boldsymbol{k}',\boldsymbol{q}')\bar{R}_{bm}^{v}(\boldsymbol{k}',\boldsymbol{q}')\\
		&=-\frac{1}{N_{\boldsymbol{q}}^3}\sum_{\boldsymbol{\delta k}\boldsymbol{k}\boldsymbol{q}'}
		\sum_{q}^{N_q}\sum_{\mu v} 
		\sum_{an}\bar{L}_{an}^{\mu *}(\boldsymbol{k},\boldsymbol{q})\bar{L}_{an}^{v}(\boldsymbol{k},\boldsymbol{q})
		\sum_{bm}\bar{R}_{bm}^{\mu *}(\boldsymbol{k}',\boldsymbol{q}')\bar{R}_{bm}^{v}(\boldsymbol{k}',\boldsymbol{q}')\\
		&=-\frac{1}{N_{\boldsymbol{q}}^3}\sum_{\boldsymbol{\delta k}\boldsymbol{k}\boldsymbol{q}'}
		\sum_{q}^{N_q}\sum_{\mu v} 
		\bar{M}_{\mu v}(\boldsymbol{k},\boldsymbol{q})\bar{N}_{\mu v}(\boldsymbol{k}',\boldsymbol{q}')
    \end{split}
\end{equation}
with 
\begin{equation}
	\begin{split}
		\bar{L}_{an}^{\mu}(\boldsymbol{k},\boldsymbol{q})
		&=w_{q}^{\frac 1 4}L_{an}^{\mu}(\boldsymbol{k},\boldsymbol{q})e^{-\frac{1}{2}t_q(\epsilon_{n\boldsymbol{q}}-\epsilon_{a\boldsymbol{k}})}\\
		\bar{R}_{bm}^{\mu}(\boldsymbol{k}',\boldsymbol{q}')
		&=w_{q}^{\frac 1 4}R_{bm}^{\mu}(\boldsymbol{k}',\boldsymbol{q}')e^{-\frac{1}{2}t_q(\epsilon_{m\boldsymbol{q}'}-\epsilon_{b\boldsymbol{k}'})}\\
		\bar{M}_{\mu v}(\boldsymbol{k},\boldsymbol{q})
		&=\sum_{an}\bar{L}_{an}^{\mu *}(\boldsymbol{k},\boldsymbol{q})\bar{L}_{an}^{v}(\boldsymbol{k},\boldsymbol{q})\\
		&=\sum_{an}w^{\frac 1 2}e^{-t_q(\epsilon_{n\boldsymbol{q}}-\epsilon_{a\boldsymbol{k}})}
		L_{an}^{\mu *}(\boldsymbol{k},\boldsymbol{q})L_{an}^{v}(\boldsymbol{k},\boldsymbol{q})\\
		\bar{N}_{\mu v}(\boldsymbol{k}',\boldsymbol{q}')
		&=\sum_{bm}\bar{R}_{bm}^{\mu *}(\boldsymbol{k}',\boldsymbol{q}')\bar{R}_{bm}^{v}(\boldsymbol{k}',\boldsymbol{q}')\\
		&=\sum_{bm}w^{\frac 1 2}e^{-t_q(\epsilon_{m\boldsymbol{q}'}-\epsilon_{b\boldsymbol{k}'})}
		R_{bm}^{\mu *}(\boldsymbol{k}',\boldsymbol{q}')R_{bm}^{v}(\boldsymbol{k}',\boldsymbol{q}')\\
	\end{split}
\end{equation}


\section{Memory distribution for periodic-PT2}

\end{document}

